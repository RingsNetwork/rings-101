\input{../_temp/temp.tex}

%% preamble
\title[Rings 101]{Rings 101 - What is Distributed Network}
% \subtitle{The subtitle}

\author[Ryan Kung]{
  \includegraphics[height=2cm]{../_temp/rings_rust.png}
  \\
  Ryan J. Kung
  \\
  Rings Network Dev Team
}

\begin{document}

%% title frame
\begin{frame}
    \titlepage
\end{frame}

\begin{frame}
  \frametitle{ What is Distributed Network }
  \begin{itemize}
  \item{ Definition: What is distributed system (1min) }
  \item{ Definition: What is a distributed Network (1min) }
  \item{ Is internet distributed? (1min) }
  \item{ Intoduce to dWeb (1min) }
  \item{ Implementation of dWeb (1min) }
  \end{itemize}
\end{frame}

\begin{frame}
  \frametitle{ Definition: What is distributed system }
  \textbf{
  \LARGE
  A system is distributed if the message transmission delay is not negligible compared to the time between events in a single process. \cite{lamport1978}
  \begin{flushright}
   --- Leslie Lamport, 1978
  \end{flushright}
  }
  \note {
  In 1978, Leslie Lamport proposed this definition and made significant contributions to the field of distributed systems, for which he was awarded the Turing Award.
  }
\end{frame}


\begin{frame}
  \frametitle{Leslie Lamport at 1978 }
  \begin{figure}[h]
  \centering
  \includegraphics[height=0.4\linewidth]{lamport.png}
  \includegraphics[width=0.5\linewidth]{lamport_paper.png}
  \end{figure}

  \note {
  In his classic paper titled "Time, Clocks, and the Ordering of Events in a Distributed System,"
  Lamport provided a description of distributed networks and introduced the "Lamport timestamp"
  as a mechanism for ordering events.
  In essence, Lamport invented logical time for the distributed world.
}
\end{frame}


\begin{frame}
  \frametitle{ Logical Timestamp }
  \begin{figure}[h]
  \centering
  \includegraphics[height=0.4\linewidth]{lamport.png}
  \includegraphics[width=0.5\linewidth]{lamport_timestamp.png}
  \end{figure}
  \note {
  The statement "message transmission delay is not negligible" in the definition of a distributed system hides an assumption that the delay in a distributed system is random and unpredictable. Therefore, given three peers A, B, and C, without any external information, C cannot independently determine the order of messages from A and B.

  The Lamport Timestamp provides a method for measuring time based on the ordering of messages, rather than relying on real-world time. This concept is extremely valuable, especially in complex distributed networks where message sequencing is essential. In the realm of cryptography, the sequencer holds paramount importance, particularly in Layer 2 implementations. We will delve deeper into this topic in future discussions.
}
\end{frame}

\begin{frame}
\frametitle{ Definition: What is distributed network }
  \begin{figure}[h]
  \centering
  \includegraphics[width=0.7\linewidth]{lopologic.png}
  \end{figure}
\end{frame}

\begin{frame}
\frametitle{ Is Internet distributed? }
\begin{columns}[T] % contents are top vertically aligned
\begin{column}{5cm} % each column can also be its own environment
  \begin{minipage}[c][0.5 \textheight][c]{\linewidth}
    \textbf{The system is failing.\cite{bernerslee2017}}
    \begin{flushright}
     --- Tim Berners-Lee, 2017
    \end{flushright}
  \end{minipage}
\end{column}
\begin{column}{5cm} % alternative top-align that's better for graphics
  \begin{figure}[h]
  \centering
  \includegraphics[width=0.7\linewidth]{hourglass.png}
  \end{figure}
\end{column}
\end{columns}
\end{frame}

\begin{frame}
\frametitle{ Intoducing to the dWeb }
  \begin{figure}[h]
  \centering
  \includegraphics[width=0.7\linewidth]{lopologic.png}\cite{mozilla2018}
  \end{figure}
\end{frame}

\begin{frame}
\frametitle{ Implementation of dWeb }
  \begin{itemize}
  \item { Tor Project }
  - Onion Relay, Hidden services
  \item { GUNnet }
  - DHT Based, structure p2p
  \item { Rings Network }
  - DHT and WASM, structure p2p
  \end{itemize}
\end{frame}

\begin{frame}
  \frametitle{Thanks}
  Repository of this Slides:\newline \\
  \centering
  \color{black}{\fcolorbox{white}{white}{
  \qrcode[link, height=1in]{https://github.com/ringsnetwork/rings-101}
  }}
\end{frame}

\begin{frame}
    \frametitle{Reference}
  \bibliographystyle{plain}
  \bibliography{slide}
\end{frame}

\end{document}